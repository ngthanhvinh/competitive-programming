\begin{problem}{Простая игра}{game.in}{game.out}{1 секунда}{256 мегабайт}

НурлашКО хорошо вел себя в прошедшем году, и за это Дед Мороз подарил ему на Новый год ломаную цепь из $N$ вершин. $i$-ая вершина ломаной находится в точке с координатами $(i, y_i)$.

Почти сразу была выпущена новая игра с этой геометрической фигурой: $M$ раз выполняется одна из следующих операций: 
\begin{itemize}  
   \item Изменить координату $y$ одной из вершин ломаной.
   \item Провести горизонтальную линию на высоте $H$ и подсчитать сколько раз она пересечется с ломаной. Отметим, что все точки горизонтальной линии имеют $y$ координату равной $H$.
\end{itemize}

НурлашКО очень понравилась эта игра и он просит Вас написать для нее программу.

\InputFile
Первая строка входного файла содержит два целых положительных числа $N, M (1 \leq N, M \leq 100\,000)$~--- количество вершин ломаной и количество операций в игре, соответственно.

Следующая строка содержит $N$ целых положительных чисел разделенных единичным пробелом $h_i (1 \leq h_i \leq 1\,000\,000)$~--- $h_i$ есть изначальная высота $i$-ой вершины.

Далее в $M$ строках идут описания операций в игре в следующем формате:
\begin{itemize}
\item $1$ $pos$ $val$ $(1 \leq pos \leq N, 1 \leq val \leq 1\,000\,000)$ --- номер вершины и новая высота, соответственно.
\item $2$ $H$ $(1 \leq H \leq 1\,000\,000)$~--- высота горизонтальной прямой. Гарантируется что эта прямая никогда не проходит через вершины ломаной.
\end{itemize}

\OutputFile
Для каждого запроса второго типа, в отдельной строке, выведите количество пересечений ломаной с горизонтальной прямой. Ответы на запросы выводите в том же порядке в котором они идут во входном файле.

\Scoring
Данная задача содержит 3 подзадачи:
\begin{enumerate}
\item $1 \leq N, M \leq 1\,000$. Оценивается в $22$ баллов.
\item $1 \leq N, M \leq 100\,000$. Разрешены только запросы (операции второго типа). Оценивается в $27$ баллов.
\item $1 \leq N, M \leq 100\,000$. Оценивается в $51$ баллов.
\end{enumerate}

Каждая следующая подзадача оценивается только при прохождении всех предыдущих.

\Example

\begin{example}
\exmp{3 3
1 5 1
2 3
1 1 5
2 3
}{2
1
}%
\end{example}

\end{problem}

